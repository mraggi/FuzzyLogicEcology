\documentclass[a4paper,12pt]{article}
\usepackage{amsmath, amssymb,amsfonts}
\usepackage[utf8]{inputenc}

\begin{document}
  Recordando que para cada especie tenemos un conjunto de puntos $A$, y para cada $a=(x_0,y_0)\in A$ tenemos
		$$f_a(x,y) = e^{-C\left[(x-x_0)^2 + (y-y_0)^2\right]}$$
  y luego
		$$\mu_A = \cup f_a$$
	
	Entonces encontramos que
	$$\mu_A = 1-\prod_{a\in A} (1-f_a)$$
	
	Encontramos que el área de méxico son como 2 millones de km$^2$. Entonces creemos que debemos tomar más o menos una ``imagen'' de 10,000$\times$10,000.
	
	Encontramos que si queremos que $\sigma = 50m$, entonces $C = 2.4\times 10^6$.
	
	\vspace{1cm}
	Ideas para acelerar los cálculos:
	\begin{itemize}
	  \item Calcular $\mu_A$ por separado para cada $A$ y obtener una imagen.
		\item A cada especie calcularle su baricentro y su radio. Cuando haya dos especies cuyos círculos no se intersectan (o que estén muy lejos), no hacer la integral.
		\item GPU?
		\item 
	\end{itemize}
	
	TODO:
	
	\begin{itemize}
 		\item Preguntar a algún ecólogo/biólogo/etc. qué es la información relevante de la red (por ejemplo, a césar).
		\item Aplicar las medidas estándar de clustering, centralidad, y clasificar la red en términos de estos números.
		\item Plotear distribución de grados, etc.
		\item Medir la convergencia en base al tamaño de la malla.
		\item Escribir paper.
	\end{itemize}


\end{document}
