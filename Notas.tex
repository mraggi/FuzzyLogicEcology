\documentclass[a4paper,12pt]{article}
\usepackage{amsmath, amssymb,amsfonts}
\usepackage[utf8]{inputenc}
\usepackage{hyperref}

\begin{document}
	\begin{center}
	  Notas para el avance del proyecto de redes.
		
		\url{https://github.com/mraggi/FuzzyLogicEcology}
	\end{center}


  Recordando que para cada especie tenemos un conjunto de puntos que se observaron $A$, y para cada $a=(x_0,y_0)\in A$ tenemos
		$$f_a(x,y) = e^{-C\left[(x-x_0)^2 + (y-y_0)^2\right]}$$
  y luego
		$$\mu_A = \cup f_a$$
	
	Todas las operaciones están con lógica difusa (o sea, la union es la suma menos el producto, etc.)
	
	\begin{itemize}
	  \item Con ppio de inclusión-exclusión, encontramos que
			$$\mu_A = 1-\prod_{a\in A} (1-f_a)$$
		\item El área de méxico son como 2 millones de km$^2$. Entonces creemos que debemos tomar más o menos una malla de 10,000$\times$10,000.
		\item Haciendo experimentos, vemos que a partir de $500\times 500$ ya empieza a variar muy muy poquito la red construida, así que probablemente no valga la pena subir hasta 10,000$\times$10,000, que tardaría varios meses en terminar, y en $1000\times 1000$ tarda como 15 segundos.
		\item Encontramos que si queremos que $\sigma = 50m$, entonces $C = 2.4\times 10^6$.
		\item Hicimos muchas cuentas para normalizar y convertir todo, poner la malla, etc.
	\end{itemize}
	
	
	
	\vspace{1cm}
	TODO:
	
	\begin{enumerate}
		\item (Miguel) Limpiar y documentar el código. Ahorita uso variables globales y cosas horribles.
		\item (Miguel + ??) Convertir el código para que use la GPU y no tarde años con mallas más grandes.
		\item (César o Víctor?) Preguntar a algún ecólogo/biólogo/etc. qué es la información relevante de la red.
		\item (??) Aplicar las medidas estándar de clustering, centralidad, y clasificar la red en términos de estos números.
		\item (??) Plottear distribución de grados, etc.
		\item (??) Escribir paper.
	\end{enumerate}


\end{document}
